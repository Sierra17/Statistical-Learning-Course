\documentclass[11pt]{article}

    \usepackage[breakable]{tcolorbox}
    \usepackage{parskip} % Stop auto-indenting (to mimic markdown behaviour)
    

    % Basic figure setup, for now with no caption control since it's done
    % automatically by Pandoc (which extracts ![](path) syntax from Markdown).
    \usepackage{graphicx}
    % Keep aspect ratio if custom image width or height is specified
    \setkeys{Gin}{keepaspectratio}
    % Maintain compatibility with old templates. Remove in nbconvert 6.0
    \let\Oldincludegraphics\includegraphics
    % Ensure that by default, figures have no caption (until we provide a
    % proper Figure object with a Caption API and a way to capture that
    % in the conversion process - todo).
    \usepackage{caption}
    \DeclareCaptionFormat{nocaption}{}
    \captionsetup{format=nocaption,aboveskip=0pt,belowskip=0pt}

    \usepackage{float}
    \floatplacement{figure}{H} % forces figures to be placed at the correct location
    \usepackage{xcolor} % Allow colors to be defined
    \usepackage{enumerate} % Needed for markdown enumerations to work
    \usepackage{geometry} % Used to adjust the document margins
    \usepackage{amsmath} % Equations
    \usepackage{amssymb} % Equations
    \usepackage{textcomp} % defines textquotesingle
    % Hack from http://tex.stackexchange.com/a/47451/13684:
    \AtBeginDocument{%
        \def\PYZsq{\textquotesingle}% Upright quotes in Pygmentized code
    }
    \usepackage{upquote} % Upright quotes for verbatim code
    \usepackage{eurosym} % defines \euro

    \usepackage{iftex}
    \ifPDFTeX
        \usepackage[T1]{fontenc}
        \IfFileExists{alphabeta.sty}{
              \usepackage{alphabeta}
          }{
              \usepackage[mathletters]{ucs}
              \usepackage[utf8x]{inputenc}
          }
    \else
        \usepackage{fontspec}
        \usepackage{unicode-math}
    \fi

    \usepackage{fancyvrb} % verbatim replacement that allows latex
    \usepackage{grffile} % extends the file name processing of package graphics
                         % to support a larger range
    \makeatletter % fix for old versions of grffile with XeLaTeX
    \@ifpackagelater{grffile}{2019/11/01}
    {
      % Do nothing on new versions
    }
    {
      \def\Gread@@xetex#1{%
        \IfFileExists{"\Gin@base".bb}%
        {\Gread@eps{\Gin@base.bb}}%
        {\Gread@@xetex@aux#1}%
      }
    }
    \makeatother
    \usepackage[Export]{adjustbox} % Used to constrain images to a maximum size
    \adjustboxset{max size={0.9\linewidth}{0.9\paperheight}}

    % The hyperref package gives us a pdf with properly built
    % internal navigation ('pdf bookmarks' for the table of contents,
    % internal cross-reference links, web links for URLs, etc.)
    \usepackage{hyperref}
    % The default LaTeX title has an obnoxious amount of whitespace. By default,
    % titling removes some of it. It also provides customization options.
    \usepackage{titling}
    \usepackage{longtable} % longtable support required by pandoc >1.10
    \usepackage{booktabs}  % table support for pandoc > 1.12.2
    \usepackage{array}     % table support for pandoc >= 2.11.3
    \usepackage{calc}      % table minipage width calculation for pandoc >= 2.11.1
    \usepackage[inline]{enumitem} % IRkernel/repr support (it uses the enumerate* environment)
    \usepackage[normalem]{ulem} % ulem is needed to support strikethroughs (\sout)
                                % normalem makes italics be italics, not underlines
    \usepackage{soul}      % strikethrough (\st) support for pandoc >= 3.0.0
    \usepackage{mathrsfs}
    

    
    % Colors for the hyperref package
    \definecolor{urlcolor}{rgb}{0,.145,.698}
    \definecolor{linkcolor}{rgb}{.71,0.21,0.01}
    \definecolor{citecolor}{rgb}{.12,.54,.11}

    % ANSI colors
    \definecolor{ansi-black}{HTML}{3E424D}
    \definecolor{ansi-black-intense}{HTML}{282C36}
    \definecolor{ansi-red}{HTML}{E75C58}
    \definecolor{ansi-red-intense}{HTML}{B22B31}
    \definecolor{ansi-green}{HTML}{00A250}
    \definecolor{ansi-green-intense}{HTML}{007427}
    \definecolor{ansi-yellow}{HTML}{DDB62B}
    \definecolor{ansi-yellow-intense}{HTML}{B27D12}
    \definecolor{ansi-blue}{HTML}{208FFB}
    \definecolor{ansi-blue-intense}{HTML}{0065CA}
    \definecolor{ansi-magenta}{HTML}{D160C4}
    \definecolor{ansi-magenta-intense}{HTML}{A03196}
    \definecolor{ansi-cyan}{HTML}{60C6C8}
    \definecolor{ansi-cyan-intense}{HTML}{258F8F}
    \definecolor{ansi-white}{HTML}{C5C1B4}
    \definecolor{ansi-white-intense}{HTML}{A1A6B2}
    \definecolor{ansi-default-inverse-fg}{HTML}{FFFFFF}
    \definecolor{ansi-default-inverse-bg}{HTML}{000000}

    % common color for the border for error outputs.
    \definecolor{outerrorbackground}{HTML}{FFDFDF}

    % commands and environments needed by pandoc snippets
    % extracted from the output of `pandoc -s`
    \providecommand{\tightlist}{%
      \setlength{\itemsep}{0pt}\setlength{\parskip}{0pt}}
    \DefineVerbatimEnvironment{Highlighting}{Verbatim}{commandchars=\\\{\}}
    % Add ',fontsize=\small' for more characters per line
    \newenvironment{Shaded}{}{}
    \newcommand{\KeywordTok}[1]{\textcolor[rgb]{0.00,0.44,0.13}{\textbf{{#1}}}}
    \newcommand{\DataTypeTok}[1]{\textcolor[rgb]{0.56,0.13,0.00}{{#1}}}
    \newcommand{\DecValTok}[1]{\textcolor[rgb]{0.25,0.63,0.44}{{#1}}}
    \newcommand{\BaseNTok}[1]{\textcolor[rgb]{0.25,0.63,0.44}{{#1}}}
    \newcommand{\FloatTok}[1]{\textcolor[rgb]{0.25,0.63,0.44}{{#1}}}
    \newcommand{\CharTok}[1]{\textcolor[rgb]{0.25,0.44,0.63}{{#1}}}
    \newcommand{\StringTok}[1]{\textcolor[rgb]{0.25,0.44,0.63}{{#1}}}
    \newcommand{\CommentTok}[1]{\textcolor[rgb]{0.38,0.63,0.69}{\textit{{#1}}}}
    \newcommand{\OtherTok}[1]{\textcolor[rgb]{0.00,0.44,0.13}{{#1}}}
    \newcommand{\AlertTok}[1]{\textcolor[rgb]{1.00,0.00,0.00}{\textbf{{#1}}}}
    \newcommand{\FunctionTok}[1]{\textcolor[rgb]{0.02,0.16,0.49}{{#1}}}
    \newcommand{\RegionMarkerTok}[1]{{#1}}
    \newcommand{\ErrorTok}[1]{\textcolor[rgb]{1.00,0.00,0.00}{\textbf{{#1}}}}
    \newcommand{\NormalTok}[1]{{#1}}

    % Additional commands for more recent versions of Pandoc
    \newcommand{\ConstantTok}[1]{\textcolor[rgb]{0.53,0.00,0.00}{{#1}}}
    \newcommand{\SpecialCharTok}[1]{\textcolor[rgb]{0.25,0.44,0.63}{{#1}}}
    \newcommand{\VerbatimStringTok}[1]{\textcolor[rgb]{0.25,0.44,0.63}{{#1}}}
    \newcommand{\SpecialStringTok}[1]{\textcolor[rgb]{0.73,0.40,0.53}{{#1}}}
    \newcommand{\ImportTok}[1]{{#1}}
    \newcommand{\DocumentationTok}[1]{\textcolor[rgb]{0.73,0.13,0.13}{\textit{{#1}}}}
    \newcommand{\AnnotationTok}[1]{\textcolor[rgb]{0.38,0.63,0.69}{\textbf{\textit{{#1}}}}}
    \newcommand{\CommentVarTok}[1]{\textcolor[rgb]{0.38,0.63,0.69}{\textbf{\textit{{#1}}}}}
    \newcommand{\VariableTok}[1]{\textcolor[rgb]{0.10,0.09,0.49}{{#1}}}
    \newcommand{\ControlFlowTok}[1]{\textcolor[rgb]{0.00,0.44,0.13}{\textbf{{#1}}}}
    \newcommand{\OperatorTok}[1]{\textcolor[rgb]{0.40,0.40,0.40}{{#1}}}
    \newcommand{\BuiltInTok}[1]{{#1}}
    \newcommand{\ExtensionTok}[1]{{#1}}
    \newcommand{\PreprocessorTok}[1]{\textcolor[rgb]{0.74,0.48,0.00}{{#1}}}
    \newcommand{\AttributeTok}[1]{\textcolor[rgb]{0.49,0.56,0.16}{{#1}}}
    \newcommand{\InformationTok}[1]{\textcolor[rgb]{0.38,0.63,0.69}{\textbf{\textit{{#1}}}}}
    \newcommand{\WarningTok}[1]{\textcolor[rgb]{0.38,0.63,0.69}{\textbf{\textit{{#1}}}}}


    % Define a nice break command that doesn't care if a line doesn't already
    % exist.
    \def\br{\hspace*{\fill} \\* }
    % Math Jax compatibility definitions
    \def\gt{>}
    \def\lt{<}
    \let\Oldtex\TeX
    \let\Oldlatex\LaTeX
    \renewcommand{\TeX}{\textrm{\Oldtex}}
    \renewcommand{\LaTeX}{\textrm{\Oldlatex}}
    % Document parameters
    % Document title
    \title{Task 2: Ensemble classifiers}
    \author{Roger Llorenç, Victor Villegas, Luis Sierra}
    
    
    
    
    
% Pygments definitions
\makeatletter
\def\PY@reset{\let\PY@it=\relax \let\PY@bf=\relax%
    \let\PY@ul=\relax \let\PY@tc=\relax%
    \let\PY@bc=\relax \let\PY@ff=\relax}
\def\PY@tok#1{\csname PY@tok@#1\endcsname}
\def\PY@toks#1+{\ifx\relax#1\empty\else%
    \PY@tok{#1}\expandafter\PY@toks\fi}
\def\PY@do#1{\PY@bc{\PY@tc{\PY@ul{%
    \PY@it{\PY@bf{\PY@ff{#1}}}}}}}
\def\PY#1#2{\PY@reset\PY@toks#1+\relax+\PY@do{#2}}

\@namedef{PY@tok@w}{\def\PY@tc##1{\textcolor[rgb]{0.73,0.73,0.73}{##1}}}
\@namedef{PY@tok@c}{\let\PY@it=\textit\def\PY@tc##1{\textcolor[rgb]{0.24,0.48,0.48}{##1}}}
\@namedef{PY@tok@cp}{\def\PY@tc##1{\textcolor[rgb]{0.61,0.40,0.00}{##1}}}
\@namedef{PY@tok@k}{\let\PY@bf=\textbf\def\PY@tc##1{\textcolor[rgb]{0.00,0.50,0.00}{##1}}}
\@namedef{PY@tok@kp}{\def\PY@tc##1{\textcolor[rgb]{0.00,0.50,0.00}{##1}}}
\@namedef{PY@tok@kt}{\def\PY@tc##1{\textcolor[rgb]{0.69,0.00,0.25}{##1}}}
\@namedef{PY@tok@o}{\def\PY@tc##1{\textcolor[rgb]{0.40,0.40,0.40}{##1}}}
\@namedef{PY@tok@ow}{\let\PY@bf=\textbf\def\PY@tc##1{\textcolor[rgb]{0.67,0.13,1.00}{##1}}}
\@namedef{PY@tok@nb}{\def\PY@tc##1{\textcolor[rgb]{0.00,0.50,0.00}{##1}}}
\@namedef{PY@tok@nf}{\def\PY@tc##1{\textcolor[rgb]{0.00,0.00,1.00}{##1}}}
\@namedef{PY@tok@nc}{\let\PY@bf=\textbf\def\PY@tc##1{\textcolor[rgb]{0.00,0.00,1.00}{##1}}}
\@namedef{PY@tok@nn}{\let\PY@bf=\textbf\def\PY@tc##1{\textcolor[rgb]{0.00,0.00,1.00}{##1}}}
\@namedef{PY@tok@ne}{\let\PY@bf=\textbf\def\PY@tc##1{\textcolor[rgb]{0.80,0.25,0.22}{##1}}}
\@namedef{PY@tok@nv}{\def\PY@tc##1{\textcolor[rgb]{0.10,0.09,0.49}{##1}}}
\@namedef{PY@tok@no}{\def\PY@tc##1{\textcolor[rgb]{0.53,0.00,0.00}{##1}}}
\@namedef{PY@tok@nl}{\def\PY@tc##1{\textcolor[rgb]{0.46,0.46,0.00}{##1}}}
\@namedef{PY@tok@ni}{\let\PY@bf=\textbf\def\PY@tc##1{\textcolor[rgb]{0.44,0.44,0.44}{##1}}}
\@namedef{PY@tok@na}{\def\PY@tc##1{\textcolor[rgb]{0.41,0.47,0.13}{##1}}}
\@namedef{PY@tok@nt}{\let\PY@bf=\textbf\def\PY@tc##1{\textcolor[rgb]{0.00,0.50,0.00}{##1}}}
\@namedef{PY@tok@nd}{\def\PY@tc##1{\textcolor[rgb]{0.67,0.13,1.00}{##1}}}
\@namedef{PY@tok@s}{\def\PY@tc##1{\textcolor[rgb]{0.73,0.13,0.13}{##1}}}
\@namedef{PY@tok@sd}{\let\PY@it=\textit\def\PY@tc##1{\textcolor[rgb]{0.73,0.13,0.13}{##1}}}
\@namedef{PY@tok@si}{\let\PY@bf=\textbf\def\PY@tc##1{\textcolor[rgb]{0.64,0.35,0.47}{##1}}}
\@namedef{PY@tok@se}{\let\PY@bf=\textbf\def\PY@tc##1{\textcolor[rgb]{0.67,0.36,0.12}{##1}}}
\@namedef{PY@tok@sr}{\def\PY@tc##1{\textcolor[rgb]{0.64,0.35,0.47}{##1}}}
\@namedef{PY@tok@ss}{\def\PY@tc##1{\textcolor[rgb]{0.10,0.09,0.49}{##1}}}
\@namedef{PY@tok@sx}{\def\PY@tc##1{\textcolor[rgb]{0.00,0.50,0.00}{##1}}}
\@namedef{PY@tok@m}{\def\PY@tc##1{\textcolor[rgb]{0.40,0.40,0.40}{##1}}}
\@namedef{PY@tok@gh}{\let\PY@bf=\textbf\def\PY@tc##1{\textcolor[rgb]{0.00,0.00,0.50}{##1}}}
\@namedef{PY@tok@gu}{\let\PY@bf=\textbf\def\PY@tc##1{\textcolor[rgb]{0.50,0.00,0.50}{##1}}}
\@namedef{PY@tok@gd}{\def\PY@tc##1{\textcolor[rgb]{0.63,0.00,0.00}{##1}}}
\@namedef{PY@tok@gi}{\def\PY@tc##1{\textcolor[rgb]{0.00,0.52,0.00}{##1}}}
\@namedef{PY@tok@gr}{\def\PY@tc##1{\textcolor[rgb]{0.89,0.00,0.00}{##1}}}
\@namedef{PY@tok@ge}{\let\PY@it=\textit}
\@namedef{PY@tok@gs}{\let\PY@bf=\textbf}
\@namedef{PY@tok@ges}{\let\PY@bf=\textbf\let\PY@it=\textit}
\@namedef{PY@tok@gp}{\let\PY@bf=\textbf\def\PY@tc##1{\textcolor[rgb]{0.00,0.00,0.50}{##1}}}
\@namedef{PY@tok@go}{\def\PY@tc##1{\textcolor[rgb]{0.44,0.44,0.44}{##1}}}
\@namedef{PY@tok@gt}{\def\PY@tc##1{\textcolor[rgb]{0.00,0.27,0.87}{##1}}}
\@namedef{PY@tok@err}{\def\PY@bc##1{{\setlength{\fboxsep}{\string -\fboxrule}\fcolorbox[rgb]{1.00,0.00,0.00}{1,1,1}{\strut ##1}}}}
\@namedef{PY@tok@kc}{\let\PY@bf=\textbf\def\PY@tc##1{\textcolor[rgb]{0.00,0.50,0.00}{##1}}}
\@namedef{PY@tok@kd}{\let\PY@bf=\textbf\def\PY@tc##1{\textcolor[rgb]{0.00,0.50,0.00}{##1}}}
\@namedef{PY@tok@kn}{\let\PY@bf=\textbf\def\PY@tc##1{\textcolor[rgb]{0.00,0.50,0.00}{##1}}}
\@namedef{PY@tok@kr}{\let\PY@bf=\textbf\def\PY@tc##1{\textcolor[rgb]{0.00,0.50,0.00}{##1}}}
\@namedef{PY@tok@bp}{\def\PY@tc##1{\textcolor[rgb]{0.00,0.50,0.00}{##1}}}
\@namedef{PY@tok@fm}{\def\PY@tc##1{\textcolor[rgb]{0.00,0.00,1.00}{##1}}}
\@namedef{PY@tok@vc}{\def\PY@tc##1{\textcolor[rgb]{0.10,0.09,0.49}{##1}}}
\@namedef{PY@tok@vg}{\def\PY@tc##1{\textcolor[rgb]{0.10,0.09,0.49}{##1}}}
\@namedef{PY@tok@vi}{\def\PY@tc##1{\textcolor[rgb]{0.10,0.09,0.49}{##1}}}
\@namedef{PY@tok@vm}{\def\PY@tc##1{\textcolor[rgb]{0.10,0.09,0.49}{##1}}}
\@namedef{PY@tok@sa}{\def\PY@tc##1{\textcolor[rgb]{0.73,0.13,0.13}{##1}}}
\@namedef{PY@tok@sb}{\def\PY@tc##1{\textcolor[rgb]{0.73,0.13,0.13}{##1}}}
\@namedef{PY@tok@sc}{\def\PY@tc##1{\textcolor[rgb]{0.73,0.13,0.13}{##1}}}
\@namedef{PY@tok@dl}{\def\PY@tc##1{\textcolor[rgb]{0.73,0.13,0.13}{##1}}}
\@namedef{PY@tok@s2}{\def\PY@tc##1{\textcolor[rgb]{0.73,0.13,0.13}{##1}}}
\@namedef{PY@tok@sh}{\def\PY@tc##1{\textcolor[rgb]{0.73,0.13,0.13}{##1}}}
\@namedef{PY@tok@s1}{\def\PY@tc##1{\textcolor[rgb]{0.73,0.13,0.13}{##1}}}
\@namedef{PY@tok@mb}{\def\PY@tc##1{\textcolor[rgb]{0.40,0.40,0.40}{##1}}}
\@namedef{PY@tok@mf}{\def\PY@tc##1{\textcolor[rgb]{0.40,0.40,0.40}{##1}}}
\@namedef{PY@tok@mh}{\def\PY@tc##1{\textcolor[rgb]{0.40,0.40,0.40}{##1}}}
\@namedef{PY@tok@mi}{\def\PY@tc##1{\textcolor[rgb]{0.40,0.40,0.40}{##1}}}
\@namedef{PY@tok@il}{\def\PY@tc##1{\textcolor[rgb]{0.40,0.40,0.40}{##1}}}
\@namedef{PY@tok@mo}{\def\PY@tc##1{\textcolor[rgb]{0.40,0.40,0.40}{##1}}}
\@namedef{PY@tok@ch}{\let\PY@it=\textit\def\PY@tc##1{\textcolor[rgb]{0.24,0.48,0.48}{##1}}}
\@namedef{PY@tok@cm}{\let\PY@it=\textit\def\PY@tc##1{\textcolor[rgb]{0.24,0.48,0.48}{##1}}}
\@namedef{PY@tok@cpf}{\let\PY@it=\textit\def\PY@tc##1{\textcolor[rgb]{0.24,0.48,0.48}{##1}}}
\@namedef{PY@tok@c1}{\let\PY@it=\textit\def\PY@tc##1{\textcolor[rgb]{0.24,0.48,0.48}{##1}}}
\@namedef{PY@tok@cs}{\let\PY@it=\textit\def\PY@tc##1{\textcolor[rgb]{0.24,0.48,0.48}{##1}}}

\def\PYZbs{\char`\\}
\def\PYZus{\char`\_}
\def\PYZob{\char`\{}
\def\PYZcb{\char`\}}
\def\PYZca{\char`\^}
\def\PYZam{\char`\&}
\def\PYZlt{\char`\<}
\def\PYZgt{\char`\>}
\def\PYZsh{\char`\#}
\def\PYZpc{\char`\%}
\def\PYZdl{\char`\$}
\def\PYZhy{\char`\-}
\def\PYZsq{\char`\'}
\def\PYZdq{\char`\"}
\def\PYZti{\char`\~}
% for compatibility with earlier versions
\def\PYZat{@}
\def\PYZlb{[}
\def\PYZrb{]}
\makeatother


    % For linebreaks inside Verbatim environment from package fancyvrb.
    \makeatletter
        \newbox\Wrappedcontinuationbox
        \newbox\Wrappedvisiblespacebox
        \newcommand*\Wrappedvisiblespace {\textcolor{red}{\textvisiblespace}}
        \newcommand*\Wrappedcontinuationsymbol {\textcolor{red}{\llap{\tiny$\m@th\hookrightarrow$}}}
        \newcommand*\Wrappedcontinuationindent {3ex }
        \newcommand*\Wrappedafterbreak {\kern\Wrappedcontinuationindent\copy\Wrappedcontinuationbox}
        % Take advantage of the already applied Pygments mark-up to insert
        % potential linebreaks for TeX processing.
        %        {, <, #, %, $, ' and ": go to next line.
        %        _, }, ^, &, >, - and ~: stay at end of broken line.
        % Use of \textquotesingle for straight quote.
        \newcommand*\Wrappedbreaksatspecials {%
            \def\PYGZus{\discretionary{\char`\_}{\Wrappedafterbreak}{\char`\_}}%
            \def\PYGZob{\discretionary{}{\Wrappedafterbreak\char`\{}{\char`\{}}%
            \def\PYGZcb{\discretionary{\char`\}}{\Wrappedafterbreak}{\char`\}}}%
            \def\PYGZca{\discretionary{\char`\^}{\Wrappedafterbreak}{\char`\^}}%
            \def\PYGZam{\discretionary{\char`\&}{\Wrappedafterbreak}{\char`\&}}%
            \def\PYGZlt{\discretionary{}{\Wrappedafterbreak\char`\<}{\char`\<}}%
            \def\PYGZgt{\discretionary{\char`\>}{\Wrappedafterbreak}{\char`\>}}%
            \def\PYGZsh{\discretionary{}{\Wrappedafterbreak\char`\#}{\char`\#}}%
            \def\PYGZpc{\discretionary{}{\Wrappedafterbreak\char`\%}{\char`\%}}%
            \def\PYGZdl{\discretionary{}{\Wrappedafterbreak\char`\$}{\char`\$}}%
            \def\PYGZhy{\discretionary{\char`\-}{\Wrappedafterbreak}{\char`\-}}%
            \def\PYGZsq{\discretionary{}{\Wrappedafterbreak\textquotesingle}{\textquotesingle}}%
            \def\PYGZdq{\discretionary{}{\Wrappedafterbreak\char`\"}{\char`\"}}%
            \def\PYGZti{\discretionary{\char`\~}{\Wrappedafterbreak}{\char`\~}}%
        }
        % Some characters . , ; ? ! / are not pygmentized.
        % This macro makes them "active" and they will insert potential linebreaks
        \newcommand*\Wrappedbreaksatpunct {%
            \lccode`\~`\.\lowercase{\def~}{\discretionary{\hbox{\char`\.}}{\Wrappedafterbreak}{\hbox{\char`\.}}}%
            \lccode`\~`\,\lowercase{\def~}{\discretionary{\hbox{\char`\,}}{\Wrappedafterbreak}{\hbox{\char`\,}}}%
            \lccode`\~`\;\lowercase{\def~}{\discretionary{\hbox{\char`\;}}{\Wrappedafterbreak}{\hbox{\char`\;}}}%
            \lccode`\~`\:\lowercase{\def~}{\discretionary{\hbox{\char`\:}}{\Wrappedafterbreak}{\hbox{\char`\:}}}%
            \lccode`\~`\?\lowercase{\def~}{\discretionary{\hbox{\char`\?}}{\Wrappedafterbreak}{\hbox{\char`\?}}}%
            \lccode`\~`\!\lowercase{\def~}{\discretionary{\hbox{\char`\!}}{\Wrappedafterbreak}{\hbox{\char`\!}}}%
            \lccode`\~`\/\lowercase{\def~}{\discretionary{\hbox{\char`\/}}{\Wrappedafterbreak}{\hbox{\char`\/}}}%
            \catcode`\.\active
            \catcode`\,\active
            \catcode`\;\active
            \catcode`\:\active
            \catcode`\?\active
            \catcode`\!\active
            \catcode`\/\active
            \lccode`\~`\~
        }
    \makeatother

    \let\OriginalVerbatim=\Verbatim
    \makeatletter
    \renewcommand{\Verbatim}[1][1]{%
        %\parskip\z@skip
        \sbox\Wrappedcontinuationbox {\Wrappedcontinuationsymbol}%
        \sbox\Wrappedvisiblespacebox {\FV@SetupFont\Wrappedvisiblespace}%
        \def\FancyVerbFormatLine ##1{\hsize\linewidth
            \vtop{\raggedright\hyphenpenalty\z@\exhyphenpenalty\z@
                \doublehyphendemerits\z@\finalhyphendemerits\z@
                \strut ##1\strut}%
        }%
        % If the linebreak is at a space, the latter will be displayed as visible
        % space at end of first line, and a continuation symbol starts next line.
        % Stretch/shrink are however usually zero for typewriter font.
        \def\FV@Space {%
            \nobreak\hskip\z@ plus\fontdimen3\font minus\fontdimen4\font
            \discretionary{\copy\Wrappedvisiblespacebox}{\Wrappedafterbreak}
            {\kern\fontdimen2\font}%
        }%

        % Allow breaks at special characters using \PYG... macros.
        \Wrappedbreaksatspecials
        % Breaks at punctuation characters . , ; ? ! and / need catcode=\active
        \OriginalVerbatim[#1,codes*=\Wrappedbreaksatpunct]%
    }
    \makeatother

    % Exact colors from NB
    \definecolor{incolor}{HTML}{303F9F}
    \definecolor{outcolor}{HTML}{D84315}
    \definecolor{cellborder}{HTML}{CFCFCF}
    \definecolor{cellbackground}{HTML}{F7F7F7}

    % prompt
    \makeatletter
    \newcommand{\boxspacing}{\kern\kvtcb@left@rule\kern\kvtcb@boxsep}
    \makeatother
    \newcommand{\prompt}[4]{
        {\ttfamily\llap{{\color{#2}[#3]:\hspace{3pt}#4}}\vspace{-\baselineskip}}
    }
    

    
    % Prevent overflowing lines due to hard-to-break entities
    \sloppy
    % Setup hyperref package
    \hypersetup{
      breaklinks=true,  % so long urls are correctly broken across lines
      colorlinks=true,
      urlcolor=urlcolor,
      linkcolor=linkcolor,
      citecolor=citecolor,
      }
    % Slightly bigger margins than the latex defaults
    
    \geometry{verbose,tmargin=1in,bmargin=1in,lmargin=1in,rmargin=1in}
    
    

\begin{document}
    
    \maketitle
    
    \tableofcontents

    
    \hypertarget{task-2-ensemble-classifiers}{%
\section{Introduction}\label{introduction}}

For this assignment we be will employing predictive models based on
decision trees to examine a dataset of cancer patients with the presence
of 102 different peptides in the patients' organisms. The data is found
in the files \texttt{cancerData.csv} and \texttt{cancerInfo.csv}.

    \begin{tcolorbox}[breakable, size=fbox, boxrule=1pt, pad at break*=1mm,colback=cellbackground, colframe=cellborder]
\prompt{In}{incolor}{1}{\boxspacing}
\begin{Verbatim}[commandchars=\\\{\}]
\PY{k+kn}{import} \PY{n+nn}{numpy} \PY{k}{as} \PY{n+nn}{np}
\PY{k+kn}{import} \PY{n+nn}{pandas} \PY{k}{as} \PY{n+nn}{pd}
\PY{k+kn}{import} \PY{n+nn}{matplotlib}\PY{n+nn}{.}\PY{n+nn}{pyplot} \PY{k}{as} \PY{n+nn}{plt}
\PY{k+kn}{import} \PY{n+nn}{sklearn}
\PY{k+kn}{from} \PY{n+nn}{sklearn}\PY{n+nn}{.}\PY{n+nn}{metrics} \PY{k+kn}{import} \PY{n}{accuracy\PYZus{}score}\PY{p}{,} \PY{n}{zero\PYZus{}one\PYZus{}loss}
\PY{k+kn}{from} \PY{n+nn}{sklearn}\PY{n+nn}{.}\PY{n+nn}{metrics} \PY{k+kn}{import} \PY{n}{confusion\PYZus{}matrix}\PY{p}{,} \PY{n}{classification\PYZus{}report}\PY{p}{,} \PY{n}{ConfusionMatrixDisplay}
\PY{k+kn}{from} \PY{n+nn}{sklearn}\PY{n+nn}{.}\PY{n+nn}{model\PYZus{}selection} \PY{k+kn}{import} \PY{n}{GridSearchCV}
\end{Verbatim}
\end{tcolorbox}

    \hypertarget{exploratory-data-analysis}{%
\section{Exploratory Data
Analysis}\label{exploratory-data-analysis}}

We first import the data and analyse its structure. We can immediately
see that the data ingestion pipeline will require some work.

    \begin{tcolorbox}[breakable, size=fbox, boxrule=1pt, pad at break*=1mm,colback=cellbackground, colframe=cellborder]
\prompt{In}{incolor}{2}{\boxspacing}
\begin{Verbatim}[commandchars=\\\{\}]
\PY{n}{df1} \PY{o}{=} \PY{n}{pd}\PY{o}{.}\PY{n}{read\PYZus{}csv}\PY{p}{(}\PY{l+s+s2}{\PYZdq{}}\PY{l+s+s2}{cancerDat.csv}\PY{l+s+s2}{\PYZdq{}}\PY{p}{,} \PY{n}{sep} \PY{o}{=} \PY{l+s+s2}{\PYZdq{}}\PY{l+s+s2}{;}\PY{l+s+s2}{\PYZdq{}}\PY{p}{,} \PY{n}{decimal} \PY{o}{=} \PY{l+s+s2}{\PYZdq{}}\PY{l+s+s2}{,}\PY{l+s+s2}{\PYZdq{}}\PY{p}{)}
\PY{n}{df1} \PY{o}{=} \PY{n}{df1}\PY{o}{.}\PY{n}{T}
\PY{n}{df1} \PY{o}{=} \PY{n}{df1}\PY{o}{.}\PY{n}{drop}\PY{p}{(}\PY{l+s+s1}{\PYZsq{}}\PY{l+s+s1}{Unnamed: 0}\PY{l+s+s1}{\PYZsq{}}\PY{p}{,} \PY{n}{axis}\PY{o}{=}\PY{l+m+mi}{0}\PY{p}{)}
\PY{n}{display}\PY{p}{(}\PY{n}{df1}\PY{o}{.}\PY{n}{head}\PY{p}{(}\PY{p}{)}\PY{p}{)}
\PY{n+nb}{print}\PY{p}{(}\PY{n}{df1}\PY{o}{.}\PY{n}{shape}\PY{p}{)}
\end{Verbatim}
\end{tcolorbox}

    
    \begin{Verbatim}[commandchars=\\\{\}]
                0          1          2          3          4          5    \textbackslash{}
NO.REC\_1  21.923472  24.442617  19.050562  18.482667  24.086793   20.32946   
NO.REC\_2  21.020165  23.649841  18.402413  19.088996  24.710323  21.495392   
NO.REC\_3  19.585788  23.736128  18.191527   16.33124  21.917326  20.284533   
NO.REC\_4  19.061767  23.374865  17.692775   15.36272  21.484924  18.379603   
NO.REC\_5  18.547029  23.039588  19.066973  15.835721  21.339587  19.550809   

                6          7          8          9    {\ldots}        92   \textbackslash{}
NO.REC\_1  19.304363  24.270429  18.878984  18.752264  {\ldots}  19.439382   
NO.REC\_2  19.454826  25.807051  19.091796  19.213397  {\ldots}  20.631064   
NO.REC\_3  16.853825  22.661125  18.215654  20.821777  {\ldots}  19.123832   
NO.REC\_4  16.513507  21.401436   18.38696  19.847221  {\ldots}  17.958307   
NO.REC\_5  16.831653  21.776832   17.85408  20.368534  {\ldots}  18.212854   

                93         94         95         96         97         98   \textbackslash{}
NO.REC\_1  16.142102  22.858297  22.262118  18.079186  17.151515  20.912124   
NO.REC\_2        NaN  22.028998  22.031468  17.101384  18.315637  21.512601   
NO.REC\_3  16.171227  23.143305  22.334392  17.159968  16.859732  19.076147   
NO.REC\_4        NaN  19.183961  17.851328  16.564709  15.161135  18.190653   
NO.REC\_5        NaN  22.228449  21.385404  17.072001  15.071656  18.896095   

                99         100        101  
NO.REC\_1  17.298159  19.097263   21.21211  
NO.REC\_2  17.100711  18.875548  23.980238  
NO.REC\_3  15.417028  16.340283  19.810886  
NO.REC\_4  15.269443  15.411408  18.351433  
NO.REC\_5  15.147357        NaN   20.28779  

[5 rows x 102 columns]
    \end{Verbatim}

    
    \begin{Verbatim}[commandchars=\\\{\}]
(129, 102)
    \end{Verbatim}

    \begin{tcolorbox}[breakable, size=fbox, boxrule=1pt, pad at break*=1mm,colback=cellbackground, colframe=cellborder]
\prompt{In}{incolor}{3}{\boxspacing}
\begin{Verbatim}[commandchars=\\\{\}]
\PY{n}{df2} \PY{o}{=} \PY{n}{pd}\PY{o}{.}\PY{n}{read\PYZus{}csv}\PY{p}{(}\PY{l+s+s2}{\PYZdq{}}\PY{l+s+s2}{cancerInfo.csv}\PY{l+s+s2}{\PYZdq{}}\PY{p}{,} \PY{n}{sep} \PY{o}{=} \PY{l+s+s2}{\PYZdq{}}\PY{l+s+s2}{;}\PY{l+s+s2}{\PYZdq{}}\PY{p}{)}
\PY{n}{df2} \PY{o}{=} \PY{n}{df2}\PY{o}{.}\PY{n}{drop}\PY{p}{(}\PY{p}{[}\PY{l+s+s1}{\PYZsq{}}\PY{l+s+s1}{Unnamed: 0}\PY{l+s+s1}{\PYZsq{}}\PY{p}{,} \PY{l+s+s1}{\PYZsq{}}\PY{l+s+s1}{Group}\PY{l+s+s1}{\PYZsq{}}\PY{p}{]}\PY{p}{,} \PY{n}{axis} \PY{o}{=} \PY{l+m+mi}{1}\PY{p}{)}
\PY{n}{df2}\PY{o}{.}\PY{n}{index} \PY{o}{=} \PY{n}{df2}\PY{o}{.}\PY{n}{iloc}\PY{p}{[}\PY{p}{:}\PY{p}{,}\PY{l+m+mi}{0}\PY{p}{]}
\PY{n}{df2} \PY{o}{=} \PY{n}{df2}\PY{o}{.}\PY{n}{drop}\PY{p}{(}\PY{p}{[}\PY{l+s+s1}{\PYZsq{}}\PY{l+s+s1}{sampleNames}\PY{l+s+s1}{\PYZsq{}}\PY{p}{]}\PY{p}{,} \PY{n}{axis} \PY{o}{=} \PY{l+m+mi}{1}\PY{p}{)}
\PY{n}{display}\PY{p}{(}\PY{n}{df2}\PY{o}{.}\PY{n}{head}\PY{p}{(}\PY{p}{)}\PY{p}{)}
\end{Verbatim}
\end{tcolorbox}

    
    \begin{Verbatim}[commandchars=\\\{\}]
            sites
sampleNames      
NO.REC\_1        A
NO.REC\_2        B
NO.REC\_3        C
NO.REC\_4        C
NO.REC\_5        C
    \end{Verbatim}

    
    We want to define a single response variable according to whether the
cancer sees a recurrence or not. This is currently coded into the key of
the patient, so we will create a separate binary vector to encode this
response, where \texttt{0} will denote non-recurrence and \texttt{1}
will denote recurrence.

    \begin{tcolorbox}[breakable, size=fbox, boxrule=1pt, pad at break*=1mm,colback=cellbackground, colframe=cellborder]
\prompt{In}{incolor}{4}{\boxspacing}
\begin{Verbatim}[commandchars=\\\{\}]
\PY{n}{indices1} \PY{o}{=} \PY{n}{np}\PY{o}{.}\PY{n}{array}\PY{p}{(}\PY{n}{df1}\PY{o}{.}\PY{n}{index}\PY{p}{)}\PY{o}{.}\PY{n}{astype}\PY{p}{(}\PY{n+nb}{str}\PY{p}{)}
\PY{n}{index\PYZus{}array1} \PY{o}{=} \PY{n}{np}\PY{o}{.}\PY{n}{zeros}\PY{p}{(}\PY{n+nb}{len}\PY{p}{(}\PY{n}{indices1}\PY{p}{)}\PY{p}{)}
\PY{n}{labels\PYZus{}1} \PY{o}{=} \PY{n}{np}\PY{o}{.}\PY{n}{where}\PY{p}{(}\PY{n}{np}\PY{o}{.}\PY{n}{char}\PY{o}{.}\PY{n}{startswith}\PY{p}{(}\PY{n}{indices1}\PY{p}{,} \PY{l+s+s1}{\PYZsq{}}\PY{l+s+s1}{NO.REC\PYZus{}}\PY{l+s+s1}{\PYZsq{}}\PY{p}{)}\PY{p}{,} \PY{l+m+mi}{0}\PY{p}{,} \PY{n}{np}\PY{o}{.}\PY{n}{where}\PY{p}{(}\PY{n}{np}\PY{o}{.}\PY{n}{char}\PY{o}{.}\PY{n}{startswith}\PY{p}{(}\PY{n}{indices1}\PY{p}{,} \PY{l+s+s1}{\PYZsq{}}\PY{l+s+s1}{REC\PYZus{}}\PY{l+s+s1}{\PYZsq{}}\PY{p}{)}\PY{p}{,} \PY{l+m+mi}{1}\PY{p}{,} \PY{n}{index\PYZus{}array1}\PY{p}{)}\PY{p}{)}

\PY{n}{indices2} \PY{o}{=} \PY{n}{np}\PY{o}{.}\PY{n}{array}\PY{p}{(}\PY{n}{df2}\PY{o}{.}\PY{n}{index}\PY{p}{)}\PY{o}{.}\PY{n}{astype}\PY{p}{(}\PY{n+nb}{str}\PY{p}{)}
\PY{n}{index\PYZus{}array2} \PY{o}{=} \PY{n}{np}\PY{o}{.}\PY{n}{zeros}\PY{p}{(}\PY{n+nb}{len}\PY{p}{(}\PY{n}{indices2}\PY{p}{)}\PY{p}{)}
\PY{n}{labels\PYZus{}2} \PY{o}{=} \PY{n}{np}\PY{o}{.}\PY{n}{where}\PY{p}{(}\PY{n}{np}\PY{o}{.}\PY{n}{char}\PY{o}{.}\PY{n}{startswith}\PY{p}{(}\PY{n}{indices2}\PY{p}{,} \PY{l+s+s1}{\PYZsq{}}\PY{l+s+s1}{NO.REC\PYZus{}}\PY{l+s+s1}{\PYZsq{}}\PY{p}{)}\PY{p}{,} \PY{l+m+mi}{0}\PY{p}{,} \PY{n}{np}\PY{o}{.}\PY{n}{where}\PY{p}{(}\PY{n}{np}\PY{o}{.}\PY{n}{char}\PY{o}{.}\PY{n}{startswith}\PY{p}{(}\PY{n}{indices2}\PY{p}{,} \PY{l+s+s1}{\PYZsq{}}\PY{l+s+s1}{REC\PYZus{}}\PY{l+s+s1}{\PYZsq{}}\PY{p}{)}\PY{p}{,} \PY{l+m+mi}{1}\PY{p}{,} \PY{n}{index\PYZus{}array2}\PY{p}{)}\PY{p}{)}

\PY{n+nb}{print}\PY{p}{(}\PY{l+s+s2}{\PYZdq{}}\PY{l+s+s2}{Labels are equal: }\PY{l+s+s2}{\PYZdq{}} \PY{o}{+} \PY{n+nb}{str}\PY{p}{(}\PY{n}{np}\PY{o}{.}\PY{n}{array\PYZus{}equal}\PY{p}{(}\PY{n}{labels\PYZus{}1}\PY{p}{,} \PY{n}{labels\PYZus{}2}\PY{p}{)}\PY{p}{)}\PY{p}{)} \PY{c+c1}{\PYZsh{} That is we can define a single response variable y for the two files}
\PY{n}{y} \PY{o}{=} \PY{n}{labels\PYZus{}1}
\end{Verbatim}
\end{tcolorbox}

    \begin{Verbatim}[commandchars=\\\{\}]
Labels are equal: True
    \end{Verbatim}

    \begin{tcolorbox}[breakable, size=fbox, boxrule=1pt, pad at break*=1mm,colback=cellbackground, colframe=cellborder]
\prompt{In}{incolor}{5}{\boxspacing}
\begin{Verbatim}[commandchars=\\\{\}]
\PY{n}{y} \PY{o}{=} \PY{n}{pd}\PY{o}{.}\PY{n}{Series}\PY{p}{(}\PY{n}{y}\PY{p}{)}
\PY{n}{display}\PY{p}{(}\PY{n}{y}\PY{p}{)}
\end{Verbatim}
\end{tcolorbox}

    
    \begin{Verbatim}[commandchars=\\\{\}]
0      0.0
1      0.0
2      0.0
3      0.0
4      0.0
      {\ldots} 
124    1.0
125    1.0
126    1.0
127    1.0
128    1.0
Length: 129, dtype: float64
    \end{Verbatim}

    
    Now we will check for missing values and some descriptive statistics of
the data. There is a substantial amount of missing values in the
dataset, so in order to avoid discarding such a large amount of
information, we resort to imputing the linear interpolation of the
missing entries. This increases the amount of information that we work
with, but we claim that for the final results this is preferable to
eliminating the entire rows and rendering our dataset almost empty.

    \begin{tcolorbox}[breakable, size=fbox, boxrule=1pt, pad at break*=1mm,colback=cellbackground, colframe=cellborder]
\prompt{In}{incolor}{6}{\boxspacing}
\begin{Verbatim}[commandchars=\\\{\}]
\PY{k}{for} \PY{n}{col} \PY{o+ow}{in} \PY{n}{df1}\PY{p}{:}
    \PY{n}{df1}\PY{p}{[}\PY{n}{col}\PY{p}{]} \PY{o}{=} \PY{n}{pd}\PY{o}{.}\PY{n}{to\PYZus{}numeric}\PY{p}{(}\PY{n}{df1}\PY{p}{[}\PY{n}{col}\PY{p}{]}\PY{p}{,} \PY{n}{errors}\PY{o}{=}\PY{l+s+s1}{\PYZsq{}}\PY{l+s+s1}{coerce}\PY{l+s+s1}{\PYZsq{}}\PY{p}{)}
\PY{n}{df1}\PY{o}{.}\PY{n}{dtypes}
\end{Verbatim}
\end{tcolorbox}

            \begin{tcolorbox}[breakable, size=fbox, boxrule=.5pt, pad at break*=1mm, opacityfill=0]
\prompt{Out}{outcolor}{6}{\boxspacing}
\begin{Verbatim}[commandchars=\\\{\}]
0      float64
1      float64
2      float64
3      float64
4      float64
        {\ldots}
97     float64
98     float64
99     float64
100    float64
101    float64
Length: 102, dtype: object
\end{Verbatim}
\end{tcolorbox}
        
    \begin{tcolorbox}[breakable, size=fbox, boxrule=1pt, pad at break*=1mm,colback=cellbackground, colframe=cellborder]
\prompt{In}{incolor}{7}{\boxspacing}
\begin{Verbatim}[commandchars=\\\{\}]
\PY{n+nb}{print}\PY{p}{(}\PY{n}{df1}\PY{o}{.}\PY{n}{isnull}\PY{p}{(}\PY{p}{)}\PY{o}{.}\PY{n}{values}\PY{o}{.}\PY{n}{any}\PY{p}{(}\PY{p}{)}\PY{p}{)}
\PY{n+nb}{print}\PY{p}{(}\PY{n+nb}{sum}\PY{p}{(}\PY{n}{df1}\PY{o}{.}\PY{n}{isnull}\PY{p}{(}\PY{p}{)}\PY{p}{)}\PY{p}{)}
\PY{n}{df1} \PY{o}{=} \PY{n}{df1}\PY{o}{.}\PY{n}{interpolate}\PY{p}{(}\PY{n}{method} \PY{o}{=}\PY{l+s+s1}{\PYZsq{}}\PY{l+s+s1}{linear}\PY{l+s+s1}{\PYZsq{}}\PY{p}{,} \PY{n}{limit\PYZus{}direction} \PY{o}{=}\PY{l+s+s1}{\PYZsq{}}\PY{l+s+s1}{forward}\PY{l+s+s1}{\PYZsq{}}\PY{p}{)}
\PY{n+nb}{print}\PY{p}{(}\PY{n}{df1}\PY{o}{.}\PY{n}{isnull}\PY{p}{(}\PY{p}{)}\PY{o}{.}\PY{n}{values}\PY{o}{.}\PY{n}{any}\PY{p}{(}\PY{p}{)}\PY{p}{)}
\end{Verbatim}
\end{tcolorbox}

    \begin{Verbatim}[commandchars=\\\{\}]
True
5151
False
    \end{Verbatim}

    \begin{tcolorbox}[breakable, size=fbox, boxrule=1pt, pad at break*=1mm,colback=cellbackground, colframe=cellborder]
\prompt{In}{incolor}{8}{\boxspacing}
\begin{Verbatim}[commandchars=\\\{\}]
\PY{n+nb}{print}\PY{p}{(}\PY{n}{df2}\PY{o}{.}\PY{n}{isnull}\PY{p}{(}\PY{p}{)}\PY{o}{.}\PY{n}{values}\PY{o}{.}\PY{n}{any}\PY{p}{(}\PY{p}{)}\PY{p}{)} \PY{c+c1}{\PYZsh{} False \PYZhy{}\PYZgt{} No nulls}
\end{Verbatim}
\end{tcolorbox}

    \begin{Verbatim}[commandchars=\\\{\}]
False
    \end{Verbatim}

    Now that we have handled the missing values, we can go ahead and perform
some more sophisticated exploration techniques.

    \begin{tcolorbox}[breakable, size=fbox, boxrule=1pt, pad at break*=1mm,colback=cellbackground, colframe=cellborder]
\prompt{In}{incolor}{9}{\boxspacing}
\begin{Verbatim}[commandchars=\\\{\}]
\PY{n}{df1}\PY{o}{.}\PY{n}{describe}\PY{p}{(}\PY{p}{)}
\end{Verbatim}
\end{tcolorbox}

            \begin{tcolorbox}[breakable, size=fbox, boxrule=.5pt, pad at break*=1mm, opacityfill=0]
\prompt{Out}{outcolor}{9}{\boxspacing}
\begin{Verbatim}[commandchars=\\\{\}]
              0           1           2           3           4           5    \textbackslash{}
count  129.000000  129.000000  129.000000  129.000000  129.000000  129.000000
mean    19.930039   23.286214   18.226328   17.331228   22.591467   20.193446
std      1.646143    0.784218    1.126192    1.477403    1.562914    1.185180
min     15.836862   21.313713   14.401524   12.976690   18.807566   16.963778
25\%     18.547029   22.755110   17.440432   16.166380   21.321595   19.397560
50\%     20.341514   23.374865   18.349787   17.373647   22.461809   20.264099
75\%     21.362745   23.793517   18.968572   18.501081   23.992847   20.961033
max     22.315946   25.387870   21.837865   19.994349   25.306821   23.671558

              6           7           8           9    {\ldots}         92   \textbackslash{}
count  129.000000  129.000000  129.000000  129.000000  {\ldots}  129.000000
mean    18.133973   23.193300   18.687522   19.516615  {\ldots}   19.126770
std      1.751196    1.860021    1.744740    1.068425  {\ldots}    1.399915
min     13.770822   18.824646   16.818849   16.952402  {\ldots}   14.488970
25\%     17.001766   22.036426   18.109188   18.830477  {\ldots}   18.323381
50\%     18.421874   23.469361   18.548956   19.546749  {\ldots}   19.240883
75\%     19.426085   24.633681   18.967436   20.108405  {\ldots}   19.956155
max     21.934433   26.596796   36.306672   22.920624  {\ldots}   23.668357

              93          94          95          96          97          98   \textbackslash{}
count  129.000000  129.000000  129.000000  129.000000  129.000000  129.000000
mean    15.832452   21.643638   21.312564   17.060424   17.096818   19.891371
std      1.005218    1.009982    1.126496    0.881310    1.241427    1.487108
min     12.398937   18.802740   17.851328   15.050081   13.746502   15.860695
25\%     15.358616   20.940953   20.538005   16.466670   16.297205   18.855134
50\%     16.099669   21.663512   21.281094   17.156167   17.270199   19.892329
75\%     16.514396   22.295952   22.004275   17.734682   18.003294   21.065003
max     19.057885   24.195671   23.932574   19.008904   20.339064   22.552429

              99          100         101
count  129.000000  129.000000  129.000000
mean    16.372295   17.653626   20.692479
std      1.011885    1.123423    2.025807
min     13.846473   14.930870   15.644922
25\%     15.707511   17.008565   19.477624
50\%     16.416639   17.691291   20.655011
75\%     17.101205   18.540819   22.009082
max     18.586913   19.888654   25.794838

[8 rows x 102 columns]
\end{Verbatim}
\end{tcolorbox}
        
    \begin{tcolorbox}[breakable, size=fbox, boxrule=1pt, pad at break*=1mm,colback=cellbackground, colframe=cellborder]
\prompt{In}{incolor}{10}{\boxspacing}
\begin{Verbatim}[commandchars=\\\{\}]
\PY{n}{df1}\PY{o}{.}\PY{n}{hist}\PY{p}{(}\PY{n}{figsize}\PY{o}{=}\PY{p}{(}\PY{l+m+mi}{12}\PY{p}{,}\PY{l+m+mi}{10}\PY{p}{)}\PY{p}{)}
\PY{n}{plt}\PY{o}{.}\PY{n}{tight\PYZus{}layout}\PY{p}{(}\PY{p}{)}
\PY{n}{plt}\PY{o}{.}\PY{n}{show}\PY{p}{(}\PY{p}{)}
\end{Verbatim}
\end{tcolorbox}

    \begin{center}
    \adjustimage{max size={0.9\linewidth}{0.9\paperheight}}{output_14_0.png}
    \end{center}
    { \hspace*{\fill} \\}
    
    In the previous cells, we can observe some erratic distributions for
some of the covariates, where they do not seem to fit very well with the
statement of the law of large numbers. Specifically, as an example, the
covariates indexed in positions \texttt{15}, \texttt{54} or \texttt{78}
have some interesting shapes.

    \begin{tcolorbox}[breakable, size=fbox, boxrule=1pt, pad at break*=1mm,colback=cellbackground, colframe=cellborder]
\prompt{In}{incolor}{11}{\boxspacing}
\begin{Verbatim}[commandchars=\\\{\}]
\PY{k+kn}{from} \PY{n+nn}{sklearn}\PY{n+nn}{.}\PY{n+nn}{decomposition} \PY{k+kn}{import} \PY{n}{PCA}
\PY{k+kn}{import} \PY{n+nn}{itertools}
\PY{n}{pca} \PY{o}{=} \PY{n}{PCA}\PY{p}{(}\PY{n}{n\PYZus{}components}\PY{o}{=}\PY{l+m+mi}{20}\PY{p}{)}
\PY{n}{pca}\PY{o}{.}\PY{n}{fit\PYZus{}transform}\PY{p}{(}\PY{n}{df1}\PY{p}{)}
\PY{c+c1}{\PYZsh{}print(pca.explained\PYZus{}variance\PYZus{}ratio\PYZus{})}
\PY{c+c1}{\PYZsh{}print(list(itertools.accumulate(pca.explained\PYZus{}variance\PYZus{}ratio\PYZus{})))}
\PY{n}{plt}\PY{o}{.}\PY{n}{plot}\PY{p}{(}\PY{n+nb}{list}\PY{p}{(}\PY{n}{itertools}\PY{o}{.}\PY{n}{accumulate}\PY{p}{(}\PY{n}{pca}\PY{o}{.}\PY{n}{explained\PYZus{}variance\PYZus{}ratio\PYZus{}}\PY{p}{)}\PY{p}{)}\PY{p}{)}
\end{Verbatim}
\end{tcolorbox}

            \begin{tcolorbox}[breakable, size=fbox, boxrule=.5pt, pad at break*=1mm, opacityfill=0]
\prompt{Out}{outcolor}{11}{\boxspacing}
\begin{Verbatim}[commandchars=\\\{\}]
[<matplotlib.lines.Line2D at 0x779719e76560>]
\end{Verbatim}
\end{tcolorbox}
        
    \begin{center}
    \adjustimage{max size={0.9\linewidth}{0.9\paperheight}}{output_16_1.png}
    \end{center}
    { \hspace*{\fill} \\}
    
    We can see from the PCA that the amount of variance accumulated by the
twenty first principal components amounts to almost 90\%. Our final
classifiers will only include ten peptides as features to train on.

    \hypertarget{train-test-split}{%
\section{Train-Test Split}\label{train-test-split}}

    \begin{tcolorbox}[breakable, size=fbox, boxrule=1pt, pad at break*=1mm,colback=cellbackground, colframe=cellborder]
\prompt{In}{incolor}{12}{\boxspacing}
\begin{Verbatim}[commandchars=\\\{\}]
\PY{k+kn}{from} \PY{n+nn}{sklearn}\PY{n+nn}{.}\PY{n+nn}{model\PYZus{}selection} \PY{k+kn}{import} \PY{n}{train\PYZus{}test\PYZus{}split}
\PY{n}{np}\PY{o}{.}\PY{n}{random}\PY{o}{.}\PY{n}{seed}\PY{p}{(}\PY{l+m+mi}{1234}\PY{p}{)}
\end{Verbatim}
\end{tcolorbox}

    We separate the data into two sets: the training set and the test set.

    \begin{tcolorbox}[breakable, size=fbox, boxrule=1pt, pad at break*=1mm,colback=cellbackground, colframe=cellborder]
\prompt{In}{incolor}{13}{\boxspacing}
\begin{Verbatim}[commandchars=\\\{\}]
\PY{n}{X\PYZus{}train}\PY{p}{,} \PY{n}{X\PYZus{}test}\PY{p}{,} \PY{n}{y\PYZus{}train}\PY{p}{,} \PY{n}{y\PYZus{}test} \PY{o}{=} \PY{n}{train\PYZus{}test\PYZus{}split}\PY{p}{(}
                                        \PY{n}{df1}\PY{p}{,}
                                        \PY{n}{y}\PY{p}{,}
                                        \PY{n}{test\PYZus{}size} \PY{o}{=} \PY{l+m+mi}{1}\PY{o}{/}\PY{l+m+mi}{3}\PY{p}{,}
                                        \PY{n}{random\PYZus{}state} \PY{o}{=} \PY{l+m+mi}{1234}\PY{p}{)}
\end{Verbatim}
\end{tcolorbox}

    \begin{tcolorbox}[breakable, size=fbox, boxrule=1pt, pad at break*=1mm,colback=cellbackground, colframe=cellborder]
\prompt{In}{incolor}{14}{\boxspacing}
\begin{Verbatim}[commandchars=\\\{\}]
\PY{n+nb}{print}\PY{p}{(}\PY{n}{X\PYZus{}train}\PY{o}{.}\PY{n}{shape}\PY{p}{,} \PY{n}{X\PYZus{}test}\PY{o}{.}\PY{n}{shape}\PY{p}{,} \PY{n}{y\PYZus{}train}\PY{o}{.}\PY{n}{shape}\PY{p}{,} \PY{n}{y\PYZus{}test}\PY{o}{.}\PY{n}{shape}\PY{p}{)}
\end{Verbatim}
\end{tcolorbox}

    \begin{Verbatim}[commandchars=\\\{\}]
(86, 102) (43, 102) (86,) (43,)
    \end{Verbatim}

    \hypertarget{model-fitting}{%
\section{Model fitting}\label{model-fitting}}

    We'll fit the following models: a Random Forest Classifier and two
Boosting Classifiers.

    \hypertarget{random-forest-classifier}{%
\subsection{Random Forest
Classifier}\label{random-forest-classifier}}

    \begin{tcolorbox}[breakable, size=fbox, boxrule=1pt, pad at break*=1mm,colback=cellbackground, colframe=cellborder]
\prompt{In}{incolor}{15}{\boxspacing}
\begin{Verbatim}[commandchars=\\\{\}]
\PY{k+kn}{from} \PY{n+nn}{sklearn}\PY{n+nn}{.}\PY{n+nn}{ensemble} \PY{k+kn}{import} \PY{n}{RandomForestClassifier}
\PY{n}{rf} \PY{o}{=} \PY{n}{RandomForestClassifier}\PY{p}{(}\PY{n}{random\PYZus{}state} \PY{o}{=} \PY{l+m+mi}{1234}\PY{p}{)}
\PY{n}{parameters} \PY{o}{=} \PY{p}{\PYZob{}}\PY{l+s+s1}{\PYZsq{}}\PY{l+s+s1}{n\PYZus{}estimators}\PY{l+s+s1}{\PYZsq{}} \PY{p}{:} \PY{n+nb}{range}\PY{p}{(}\PY{l+m+mi}{10}\PY{p}{,}\PY{l+m+mi}{130}\PY{p}{,}\PY{l+m+mi}{20}\PY{p}{)}\PY{p}{,}
              \PY{l+s+s1}{\PYZsq{}}\PY{l+s+s1}{max\PYZus{}features}\PY{l+s+s1}{\PYZsq{}} \PY{p}{:} \PY{p}{(}\PY{l+m+mi}{1}\PY{p}{,}\PY{l+m+mi}{2}\PY{p}{,}\PY{l+m+mi}{3}\PY{p}{,}\PY{l+m+mi}{4}\PY{p}{,}\PY{l+m+mi}{5}\PY{p}{,}\PY{l+m+mi}{6}\PY{p}{,}\PY{l+m+mi}{7}\PY{p}{,}\PY{l+m+mi}{8}\PY{p}{,}\PY{l+m+mi}{9}\PY{p}{,}\PY{l+m+mi}{10}\PY{p}{)}
             \PY{p}{\PYZcb{}}
\end{Verbatim}
\end{tcolorbox}

    We implement a grid search procedure.

    \begin{tcolorbox}[breakable, size=fbox, boxrule=1pt, pad at break*=1mm,colback=cellbackground, colframe=cellborder]
\prompt{In}{incolor}{16}{\boxspacing}
\begin{Verbatim}[commandchars=\\\{\}]
\PY{n}{grid\PYZus{}rf} \PY{o}{=} \PY{n}{GridSearchCV}\PY{p}{(}\PY{n}{rf}\PY{p}{,} \PY{n}{parameters}\PY{p}{)}
\PY{n}{grid\PYZus{}rf}\PY{o}{.}\PY{n}{fit}\PY{p}{(}\PY{n}{X\PYZus{}train}\PY{p}{,} \PY{n}{y\PYZus{}train}\PY{p}{)}
\end{Verbatim}
\end{tcolorbox}

            \begin{tcolorbox}[breakable, size=fbox, boxrule=.5pt, pad at break*=1mm, opacityfill=0]
\prompt{Out}{outcolor}{16}{\boxspacing}
\begin{Verbatim}[commandchars=\\\{\}]
GridSearchCV(estimator=RandomForestClassifier(random\_state=1234),
             param\_grid=\{'max\_features': (1, 2, 3, 4, 5, 6, 7, 8, 9, 10),
                         'n\_estimators': range(10, 130, 20)\})
\end{Verbatim}
\end{tcolorbox}
        
    \begin{tcolorbox}[breakable, size=fbox, boxrule=1pt, pad at break*=1mm,colback=cellbackground, colframe=cellborder]
\prompt{In}{incolor}{17}{\boxspacing}
\begin{Verbatim}[commandchars=\\\{\}]
\PY{n}{rf\PYZus{}results} \PY{o}{=} \PY{n}{pd}\PY{o}{.}\PY{n}{DataFrame}\PY{p}{(}\PY{n}{grid\PYZus{}rf}\PY{o}{.}\PY{n}{cv\PYZus{}results\PYZus{}}\PY{p}{)}
\PY{n}{rf\PYZus{}results}\PY{o}{.}\PY{n}{filter}\PY{p}{(}\PY{n}{regex} \PY{o}{=} \PY{l+s+s1}{\PYZsq{}}\PY{l+s+s1}{(param.*|mean\PYZus{}t|std\PYZus{}t)}\PY{l+s+s1}{\PYZsq{}}\PY{p}{)} \PYZbs{}
    \PY{o}{.}\PY{n}{drop}\PY{p}{(}\PY{n}{columns} \PY{o}{=} \PY{l+s+s1}{\PYZsq{}}\PY{l+s+s1}{params}\PY{l+s+s1}{\PYZsq{}}\PY{p}{)} \PYZbs{}
    \PY{o}{.}\PY{n}{sort\PYZus{}values}\PY{p}{(}\PY{l+s+s1}{\PYZsq{}}\PY{l+s+s1}{mean\PYZus{}test\PYZus{}score}\PY{l+s+s1}{\PYZsq{}}\PY{p}{,} \PY{n}{ascending} \PY{o}{=} \PY{k+kc}{False}\PY{p}{)} \PYZbs{}
    \PY{o}{.}\PY{n}{head}\PY{p}{(}\PY{l+m+mi}{4}\PY{p}{)}
\end{Verbatim}
\end{tcolorbox}

            \begin{tcolorbox}[breakable, size=fbox, boxrule=.5pt, pad at break*=1mm, opacityfill=0]
\prompt{Out}{outcolor}{17}{\boxspacing}
\begin{Verbatim}[commandchars=\\\{\}]
   param\_max\_features param\_n\_estimators  mean\_test\_score  std\_test\_score
36                  7                 10         0.594118        0.118234
55                 10                 30         0.582353        0.079792
48                  9                 10         0.572549        0.142693
1                   1                 30         0.569935        0.044213
\end{Verbatim}
\end{tcolorbox}
        
    \begin{tcolorbox}[breakable, size=fbox, boxrule=1pt, pad at break*=1mm,colback=cellbackground, colframe=cellborder]
\prompt{In}{incolor}{18}{\boxspacing}
\begin{Verbatim}[commandchars=\\\{\}]
\PY{n}{best\PYZus{}params} \PY{o}{=} \PY{n}{grid\PYZus{}rf}\PY{o}{.}\PY{n}{best\PYZus{}params\PYZus{}}
\PY{n+nb}{print}\PY{p}{(}\PY{l+s+s2}{\PYZdq{}}\PY{l+s+s2}{Best Parameters:}\PY{l+s+s2}{\PYZdq{}}\PY{p}{,} \PY{n}{best\PYZus{}params}\PY{p}{)}
\PY{n}{best\PYZus{}rf} \PY{o}{=} \PY{n}{RandomForestClassifier}\PY{p}{(}\PY{o}{*}\PY{o}{*}\PY{n}{best\PYZus{}params}\PY{p}{,} \PY{n}{random\PYZus{}state}\PY{o}{=}\PY{l+m+mi}{1234}\PY{p}{)}
\PY{n}{best\PYZus{}rf}\PY{o}{.}\PY{n}{fit}\PY{p}{(}\PY{n}{X\PYZus{}train}\PY{p}{,} \PY{n}{y\PYZus{}train}\PY{p}{)}
\end{Verbatim}
\end{tcolorbox}

    \begin{Verbatim}[commandchars=\\\{\}]
Best Parameters: \{'max\_features': 7, 'n\_estimators': 10\}
    \end{Verbatim}

            \begin{tcolorbox}[breakable, size=fbox, boxrule=.5pt, pad at break*=1mm, opacityfill=0]
\prompt{Out}{outcolor}{18}{\boxspacing}
\begin{Verbatim}[commandchars=\\\{\}]
RandomForestClassifier(max\_features=7, n\_estimators=10, random\_state=1234)
\end{Verbatim}
\end{tcolorbox}
        
    The accuracy and performance of RF is as follows.

    \begin{tcolorbox}[breakable, size=fbox, boxrule=1pt, pad at break*=1mm,colback=cellbackground, colframe=cellborder]
\prompt{In}{incolor}{19}{\boxspacing}
\begin{Verbatim}[commandchars=\\\{\}]
\PY{n}{y\PYZus{}pred} \PY{o}{=} \PY{n}{best\PYZus{}rf}\PY{o}{.}\PY{n}{predict}\PY{p}{(}\PY{n}{X\PYZus{}test}\PY{p}{)}
\PY{n}{accuracy\PYZus{}score}\PY{p}{(}\PY{n}{y\PYZus{}pred}\PY{p}{,} \PY{n}{y\PYZus{}test}\PY{p}{)}
\end{Verbatim}
\end{tcolorbox}

            \begin{tcolorbox}[breakable, size=fbox, boxrule=.5pt, pad at break*=1mm, opacityfill=0]
\prompt{Out}{outcolor}{19}{\boxspacing}
\begin{Verbatim}[commandchars=\\\{\}]
0.5348837209302325
\end{Verbatim}
\end{tcolorbox}
        
    \begin{tcolorbox}[breakable, size=fbox, boxrule=1pt, pad at break*=1mm,colback=cellbackground, colframe=cellborder]
\prompt{In}{incolor}{20}{\boxspacing}
\begin{Verbatim}[commandchars=\\\{\}]
\PY{n}{confusion\PYZus{}mtx} \PY{o}{=} \PY{n}{confusion\PYZus{}matrix}\PY{p}{(}\PY{n}{y\PYZus{}test}\PY{p}{,} \PY{n}{y\PYZus{}pred}\PY{p}{)} 
\PY{n}{disp} \PY{o}{=} \PY{n}{ConfusionMatrixDisplay}\PY{p}{(}\PY{n}{confusion\PYZus{}matrix}\PY{o}{=} \PY{n}{confusion\PYZus{}mtx}\PY{p}{,} \PY{n}{display\PYZus{}labels}\PY{o}{=} \PY{k+kc}{None}\PY{p}{)}
\PY{n}{disp}\PY{o}{.}\PY{n}{plot}\PY{p}{(}\PY{n}{cmap}\PY{o}{=} \PY{n}{plt}\PY{o}{.}\PY{n}{cm}\PY{o}{.}\PY{n}{Blues}\PY{p}{)}
\PY{n}{plt}\PY{o}{.}\PY{n}{title}\PY{p}{(}\PY{l+s+s2}{\PYZdq{}}\PY{l+s+s2}{Confusion matrix Random Forest}\PY{l+s+s2}{\PYZdq{}}\PY{p}{)}
\PY{n}{plt}\PY{o}{.}\PY{n}{show}\PY{p}{(}\PY{p}{)}
\end{Verbatim}
\end{tcolorbox}

    \begin{center}
    \adjustimage{max size={0.9\linewidth}{0.9\paperheight}}{output_33_0.png}
    \end{center}
    { \hspace*{\fill} \\}
    
    Finally, the most relevant variables in the prediction are given by the
following code:

    \begin{tcolorbox}[breakable, size=fbox, boxrule=1pt, pad at break*=1mm,colback=cellbackground, colframe=cellborder]
\prompt{In}{incolor}{21}{\boxspacing}
\begin{Verbatim}[commandchars=\\\{\}]
\PY{n}{feature\PYZus{}importances} \PY{o}{=} \PY{n}{best\PYZus{}rf}\PY{o}{.}\PY{n}{feature\PYZus{}importances\PYZus{}}
\PY{n}{importance\PYZus{}df} \PY{o}{=} \PY{n}{pd}\PY{o}{.}\PY{n}{DataFrame}\PY{p}{(}\PY{p}{\PYZob{}}\PY{l+s+s1}{\PYZsq{}}\PY{l+s+s1}{Feature}\PY{l+s+s1}{\PYZsq{}} \PY{p}{:} \PY{n}{X\PYZus{}train}\PY{o}{.}\PY{n}{columns}\PY{p}{,} \PY{l+s+s1}{\PYZsq{}}\PY{l+s+s1}{Importance}\PY{l+s+s1}{\PYZsq{}} \PY{p}{:} \PY{n}{feature\PYZus{}importances}\PY{p}{\PYZcb{}}\PY{p}{)}
\PY{n+nb}{print}\PY{p}{(}\PY{n}{importance\PYZus{}df}\PY{o}{.}\PY{n}{sort\PYZus{}values}\PY{p}{(}\PY{n}{by}\PY{o}{=}\PY{p}{[}\PY{l+s+s1}{\PYZsq{}}\PY{l+s+s1}{Importance}\PY{l+s+s1}{\PYZsq{}}\PY{p}{]}\PY{p}{,} \PY{n}{ascending}\PY{o}{=}\PY{k+kc}{False}\PY{p}{)}\PY{o}{.}\PY{n}{head}\PY{p}{(}\PY{l+m+mi}{10}\PY{p}{)}\PY{p}{)}
\end{Verbatim}
\end{tcolorbox}

    \begin{Verbatim}[commandchars=\\\{\}]
    Feature  Importance
51       51    0.057151
11       11    0.043937
25       25    0.032852
96       96    0.030232
78       78    0.029687
76       76    0.028671
5         5    0.027653
34       34    0.026475
12       12    0.026456
62       62    0.024448
    \end{Verbatim}

    \hypertarget{gradient-boosting}{%
\subsection{Gradient Boosting}\label{gradient-boosting}}

    Using stumps as classification trees for the response variable, we
compute the misclassification rates of both the learning set and the
test set across 2,000 iterations.

    \begin{tcolorbox}[breakable, size=fbox, boxrule=1pt, pad at break*=1mm,colback=cellbackground, colframe=cellborder]
\prompt{In}{incolor}{22}{\boxspacing}
\begin{Verbatim}[commandchars=\\\{\}]
\PY{k+kn}{from} \PY{n+nn}{sklearn}\PY{n+nn}{.}\PY{n+nn}{ensemble} \PY{k+kn}{import} \PY{n}{GradientBoostingClassifier}
\PY{n}{gb} \PY{o}{=} \PY{n}{GradientBoostingClassifier}\PY{p}{(}\PY{n}{n\PYZus{}estimators}\PY{o}{=}\PY{l+m+mi}{2000}\PY{p}{,} \PY{n}{random\PYZus{}state}\PY{o}{=}\PY{l+m+mi}{1234}\PY{p}{)}
\PY{n}{parameters} \PY{o}{=} \PY{p}{\PYZob{}}\PY{l+s+s1}{\PYZsq{}}\PY{l+s+s1}{max\PYZus{}depth}\PY{l+s+s1}{\PYZsq{}} \PY{p}{:} \PY{p}{(}\PY{l+m+mi}{1}\PY{p}{,}\PY{l+m+mi}{4}\PY{p}{,}\PY{l+m+mi}{8}\PY{p}{,}\PY{l+m+mi}{16}\PY{p}{)}\PY{p}{\PYZcb{}}
\end{Verbatim}
\end{tcolorbox}

    \begin{tcolorbox}[breakable, size=fbox, boxrule=1pt, pad at break*=1mm,colback=cellbackground, colframe=cellborder]
\prompt{In}{incolor}{23}{\boxspacing}
\begin{Verbatim}[commandchars=\\\{\}]
\PY{n}{grid\PYZus{}gb} \PY{o}{=} \PY{n}{GridSearchCV}\PY{p}{(}\PY{n}{gb}\PY{p}{,} \PY{n}{parameters}\PY{p}{)}
\PY{n}{grid\PYZus{}gb}\PY{o}{.}\PY{n}{fit}\PY{p}{(}\PY{n}{X\PYZus{}train}\PY{p}{,} \PY{n}{y\PYZus{}train}\PY{p}{)}
\end{Verbatim}
\end{tcolorbox}

            \begin{tcolorbox}[breakable, size=fbox, boxrule=.5pt, pad at break*=1mm, opacityfill=0]
\prompt{Out}{outcolor}{23}{\boxspacing}
\begin{Verbatim}[commandchars=\\\{\}]
GridSearchCV(estimator=GradientBoostingClassifier(n\_estimators=2000,
                                                  random\_state=1234),
             param\_grid=\{'max\_depth': (1, 4, 8, 16)\})
\end{Verbatim}
\end{tcolorbox}
        
    The results of comparing the test-set misclassification rates attained
by different ensemble classifiers based on trees with varying maximum
depth is shown in the following table.

    \begin{tcolorbox}[breakable, size=fbox, boxrule=1pt, pad at break*=1mm,colback=cellbackground, colframe=cellborder]
\prompt{In}{incolor}{24}{\boxspacing}
\begin{Verbatim}[commandchars=\\\{\}]
\PY{n}{gb\PYZus{}results} \PY{o}{=} \PY{n}{pd}\PY{o}{.}\PY{n}{DataFrame}\PY{p}{(}\PY{n}{grid\PYZus{}gb}\PY{o}{.}\PY{n}{cv\PYZus{}results\PYZus{}}\PY{p}{)}
\PY{n}{gb\PYZus{}results}\PY{o}{.}\PY{n}{filter}\PY{p}{(}\PY{n}{regex} \PY{o}{=} \PY{l+s+s1}{\PYZsq{}}\PY{l+s+s1}{(param.*|mean\PYZus{}t|std\PYZus{}t)}\PY{l+s+s1}{\PYZsq{}}\PY{p}{)} \PYZbs{}
    \PY{o}{.}\PY{n}{drop}\PY{p}{(}\PY{n}{columns} \PY{o}{=} \PY{l+s+s1}{\PYZsq{}}\PY{l+s+s1}{params}\PY{l+s+s1}{\PYZsq{}}\PY{p}{)} \PYZbs{}
    \PY{o}{.}\PY{n}{sort\PYZus{}values}\PY{p}{(}\PY{l+s+s1}{\PYZsq{}}\PY{l+s+s1}{mean\PYZus{}test\PYZus{}score}\PY{l+s+s1}{\PYZsq{}}\PY{p}{,} \PY{n}{ascending} \PY{o}{=} \PY{k+kc}{False}\PY{p}{)} \PYZbs{}
    \PY{o}{.}\PY{n}{head}\PY{p}{(}\PY{l+m+mi}{4}\PY{p}{)}
\end{Verbatim}
\end{tcolorbox}

            \begin{tcolorbox}[breakable, size=fbox, boxrule=.5pt, pad at break*=1mm, opacityfill=0]
\prompt{Out}{outcolor}{24}{\boxspacing}
\begin{Verbatim}[commandchars=\\\{\}]
  param\_max\_depth  mean\_test\_score  std\_test\_score
0               1         0.524183        0.069834
2               8         0.454248        0.103714
3              16         0.454248        0.103714
1               4         0.442484        0.090319
\end{Verbatim}
\end{tcolorbox}
        
    The following plot represents graphically the error as a function of the
number of boosting iterations.

    \begin{tcolorbox}[breakable, size=fbox, boxrule=1pt, pad at break*=1mm,colback=cellbackground, colframe=cellborder]
\prompt{In}{incolor}{25}{\boxspacing}
\begin{Verbatim}[commandchars=\\\{\}]
\PY{n}{best\PYZus{}params} \PY{o}{=} \PY{n}{grid\PYZus{}gb}\PY{o}{.}\PY{n}{best\PYZus{}params\PYZus{}}
\PY{n+nb}{print}\PY{p}{(}\PY{l+s+s2}{\PYZdq{}}\PY{l+s+s2}{Best Parameters:}\PY{l+s+s2}{\PYZdq{}}\PY{p}{,} \PY{n}{best\PYZus{}params}\PY{p}{)}

\PY{n}{train\PYZus{}errors} \PY{o}{=} \PY{p}{[}\PY{p}{]}
\PY{n}{best\PYZus{}gb} \PY{o}{=} \PY{n}{GradientBoostingClassifier}\PY{p}{(}\PY{o}{*}\PY{o}{*}\PY{n}{best\PYZus{}params}\PY{p}{,} \PY{n}{random\PYZus{}state}\PY{o}{=}\PY{l+m+mi}{1234}\PY{p}{)}
\PY{n}{best\PYZus{}gb}\PY{o}{.}\PY{n}{fit}\PY{p}{(}\PY{n}{X\PYZus{}train}\PY{p}{,} \PY{n}{y\PYZus{}train}\PY{p}{)}

\PY{k}{for} \PY{n}{i}\PY{p}{,} \PY{n}{y\PYZus{}pred\PYZus{}train} \PY{o+ow}{in} \PY{n+nb}{enumerate}\PY{p}{(}\PY{n}{best\PYZus{}gb}\PY{o}{.}\PY{n}{staged\PYZus{}predict}\PY{p}{(}\PY{n}{X\PYZus{}train}\PY{p}{)}\PY{p}{)}\PY{p}{:}
    \PY{n}{train\PYZus{}errors}\PY{o}{.}\PY{n}{append}\PY{p}{(}\PY{n}{zero\PYZus{}one\PYZus{}loss}\PY{p}{(}\PY{n}{y\PYZus{}train}\PY{p}{,} \PY{n}{y\PYZus{}pred\PYZus{}train}\PY{p}{)}\PY{p}{)}

\PY{n}{plt}\PY{o}{.}\PY{n}{plot}\PY{p}{(}\PY{n}{np}\PY{o}{.}\PY{n}{arange}\PY{p}{(}\PY{l+m+mi}{0}\PY{p}{,}\PY{l+m+mi}{100}\PY{p}{)}\PY{p}{,} \PY{n}{train\PYZus{}errors}\PY{p}{)}
\PY{n}{plt}\PY{o}{.}\PY{n}{xlabel}\PY{p}{(}\PY{l+s+s1}{\PYZsq{}}\PY{l+s+s1}{Number of Boosting iterations}\PY{l+s+s1}{\PYZsq{}}\PY{p}{)}
\PY{n}{plt}\PY{o}{.}\PY{n}{ylabel}\PY{p}{(}\PY{l+s+s1}{\PYZsq{}}\PY{l+s+s1}{Misclassification rate}\PY{l+s+s1}{\PYZsq{}}\PY{p}{)}
\PY{n}{plt}\PY{o}{.}\PY{n}{title}\PY{p}{(}\PY{l+s+s1}{\PYZsq{}}\PY{l+s+s1}{Gradient Boosting Error Evolution}\PY{l+s+s1}{\PYZsq{}}\PY{p}{)}
\end{Verbatim}
\end{tcolorbox}

    \begin{Verbatim}[commandchars=\\\{\}]
Best Parameters: \{'max\_depth': 1\}
    \end{Verbatim}

            \begin{tcolorbox}[breakable, size=fbox, boxrule=.5pt, pad at break*=1mm, opacityfill=0]
\prompt{Out}{outcolor}{25}{\boxspacing}
\begin{Verbatim}[commandchars=\\\{\}]
Text(0.5, 1.0, 'Gradient Boosting Error Evolution')
\end{Verbatim}
\end{tcolorbox}
        
    \begin{center}
    \adjustimage{max size={0.9\linewidth}{0.9\paperheight}}{output_43_2.png}
    \end{center}
    { \hspace*{\fill} \\}
    
    The accuracy and performance of Gradient Boosting is as follows.

    \begin{tcolorbox}[breakable, size=fbox, boxrule=1pt, pad at break*=1mm,colback=cellbackground, colframe=cellborder]
\prompt{In}{incolor}{34}{\boxspacing}
\begin{Verbatim}[commandchars=\\\{\}]
\PY{n}{y\PYZus{}pred} \PY{o}{=} \PY{n}{best\PYZus{}gb}\PY{o}{.}\PY{n}{predict}\PY{p}{(}\PY{n}{X\PYZus{}test}\PY{p}{)}
\PY{n}{accuracy\PYZus{}score}\PY{p}{(}\PY{n}{y\PYZus{}pred}\PY{p}{,} \PY{n}{y\PYZus{}test}\PY{p}{)}
\end{Verbatim}
\end{tcolorbox}

            \begin{tcolorbox}[breakable, size=fbox, boxrule=.5pt, pad at break*=1mm, opacityfill=0]
\prompt{Out}{outcolor}{34}{\boxspacing}
\begin{Verbatim}[commandchars=\\\{\}]
0.627906976744186
\end{Verbatim}
\end{tcolorbox}
        
    \begin{tcolorbox}[breakable, size=fbox, boxrule=1pt, pad at break*=1mm,colback=cellbackground, colframe=cellborder]
\prompt{In}{incolor}{35}{\boxspacing}
\begin{Verbatim}[commandchars=\\\{\}]
\PY{n}{confusion\PYZus{}mtx} \PY{o}{=} \PY{n}{confusion\PYZus{}matrix}\PY{p}{(}\PY{n}{y\PYZus{}test}\PY{p}{,} \PY{n}{y\PYZus{}pred}\PY{p}{)} 
\PY{n}{disp} \PY{o}{=} \PY{n}{ConfusionMatrixDisplay}\PY{p}{(}\PY{n}{confusion\PYZus{}matrix}\PY{o}{=} \PY{n}{confusion\PYZus{}mtx}\PY{p}{,} \PY{n}{display\PYZus{}labels}\PY{o}{=} \PY{k+kc}{None}\PY{p}{)}
\PY{n}{disp}\PY{o}{.}\PY{n}{plot}\PY{p}{(}\PY{n}{cmap}\PY{o}{=} \PY{n}{plt}\PY{o}{.}\PY{n}{cm}\PY{o}{.}\PY{n}{Blues}\PY{p}{)}
\PY{n}{plt}\PY{o}{.}\PY{n}{title}\PY{p}{(}\PY{l+s+s2}{\PYZdq{}}\PY{l+s+s2}{Confusion matrix Gradient Boosting}\PY{l+s+s2}{\PYZdq{}}\PY{p}{)}
\PY{n}{plt}\PY{o}{.}\PY{n}{show}\PY{p}{(}\PY{p}{)}
\end{Verbatim}
\end{tcolorbox}

    \begin{center}
    \adjustimage{max size={0.9\linewidth}{0.9\paperheight}}{output_46_0.png}
    \end{center}
    { \hspace*{\fill} \\}
    
    \hypertarget{xgboost}{%
\subsection{XGBoost}\label{xgboost}}

    To conclude, we introduce another boosting flavour and analyse its
performance.

    \begin{tcolorbox}[breakable, size=fbox, boxrule=1pt, pad at break*=1mm,colback=cellbackground, colframe=cellborder]
\prompt{In}{incolor}{36}{\boxspacing}
\begin{Verbatim}[commandchars=\\\{\}]
\PY{k+kn}{import} \PY{n+nn}{xgboost}
\PY{n}{xgb} \PY{o}{=} \PY{n}{xgboost}\PY{o}{.}\PY{n}{XGBClassifier}\PY{p}{(}\PY{n}{n\PYZus{}estimators}\PY{o}{=}\PY{l+m+mi}{2000}\PY{p}{,} \PY{n}{random\PYZus{}state}\PY{o}{=}\PY{l+m+mi}{1234}\PY{p}{)}
\PY{n}{parameters} \PY{o}{=} \PY{p}{\PYZob{}}\PY{l+s+s1}{\PYZsq{}}\PY{l+s+s1}{max\PYZus{}depth}\PY{l+s+s1}{\PYZsq{}} \PY{p}{:} \PY{p}{(}\PY{l+m+mi}{1}\PY{p}{,}\PY{l+m+mi}{4}\PY{p}{,}\PY{l+m+mi}{8}\PY{p}{,}\PY{l+m+mi}{16}\PY{p}{)}\PY{p}{,}
             \PY{l+s+s1}{\PYZsq{}}\PY{l+s+s1}{learning\PYZus{}rate}\PY{l+s+s1}{\PYZsq{}} \PY{p}{:} \PY{p}{[}\PY{l+m+mi}{0}\PY{p}{,}\PY{l+m+mi}{2}\PY{p}{]}
             \PY{p}{\PYZcb{}}
\end{Verbatim}
\end{tcolorbox}

    \begin{tcolorbox}[breakable, size=fbox, boxrule=1pt, pad at break*=1mm,colback=cellbackground, colframe=cellborder]
\prompt{In}{incolor}{37}{\boxspacing}
\begin{Verbatim}[commandchars=\\\{\}]
\PY{n}{grid\PYZus{}xgb} \PY{o}{=} \PY{n}{GridSearchCV}\PY{p}{(}\PY{n}{xgb}\PY{p}{,} \PY{n}{parameters}\PY{p}{)}
\PY{n}{grid\PYZus{}xgb}\PY{o}{.}\PY{n}{fit}\PY{p}{(}\PY{n}{X\PYZus{}train}\PY{p}{,} \PY{n}{y\PYZus{}train}\PY{p}{)}
\end{Verbatim}
\end{tcolorbox}

            \begin{tcolorbox}[breakable, size=fbox, boxrule=.5pt, pad at break*=1mm, opacityfill=0]
\prompt{Out}{outcolor}{37}{\boxspacing}
\begin{Verbatim}[commandchars=\\\{\}]
GridSearchCV(estimator=XGBClassifier(base\_score=None, booster=None,
                                     callbacks=None, colsample\_bylevel=None,
                                     colsample\_bynode=None,
                                     colsample\_bytree=None, device=None,
                                     early\_stopping\_rounds=None,
                                     enable\_categorical=False, eval\_metric=None,
                                     feature\_types=None, gamma=None,
                                     grow\_policy=None, importance\_type=None,
                                     interaction\_constraints=None,
                                     learning\_rate=None, max\_bin=None,
                                     max\_cat\_threshold=None,
                                     max\_cat\_to\_onehot=None,
                                     max\_delta\_step=None, max\_depth=None,
                                     max\_leaves=None, min\_child\_weight=None,
                                     missing=nan, monotone\_constraints=None,
                                     multi\_strategy=None, n\_estimators=2000,
                                     n\_jobs=None, num\_parallel\_tree=None,
                                     random\_state=1234, {\ldots}),
             param\_grid=\{'learning\_rate': [0, 2], 'max\_depth': (1, 4, 8, 16)\})
\end{Verbatim}
\end{tcolorbox}
        
    \begin{tcolorbox}[breakable, size=fbox, boxrule=1pt, pad at break*=1mm,colback=cellbackground, colframe=cellborder]
\prompt{In}{incolor}{38}{\boxspacing}
\begin{Verbatim}[commandchars=\\\{\}]
\PY{n}{xgb\PYZus{}results} \PY{o}{=} \PY{n}{pd}\PY{o}{.}\PY{n}{DataFrame}\PY{p}{(}\PY{n}{grid\PYZus{}xgb}\PY{o}{.}\PY{n}{cv\PYZus{}results\PYZus{}}\PY{p}{)}
\PY{n}{xgb\PYZus{}results}\PY{o}{.}\PY{n}{filter}\PY{p}{(}\PY{n}{regex} \PY{o}{=} \PY{l+s+s1}{\PYZsq{}}\PY{l+s+s1}{(param.*|mean\PYZus{}t|std\PYZus{}t)}\PY{l+s+s1}{\PYZsq{}}\PY{p}{)} \PYZbs{}
    \PY{o}{.}\PY{n}{drop}\PY{p}{(}\PY{n}{columns} \PY{o}{=} \PY{l+s+s1}{\PYZsq{}}\PY{l+s+s1}{params}\PY{l+s+s1}{\PYZsq{}}\PY{p}{)} \PYZbs{}
    \PY{o}{.}\PY{n}{sort\PYZus{}values}\PY{p}{(}\PY{l+s+s1}{\PYZsq{}}\PY{l+s+s1}{mean\PYZus{}test\PYZus{}score}\PY{l+s+s1}{\PYZsq{}}\PY{p}{,} \PY{n}{ascending} \PY{o}{=} \PY{k+kc}{False}\PY{p}{)} \PYZbs{}
    \PY{o}{.}\PY{n}{head}\PY{p}{(}\PY{l+m+mi}{4}\PY{p}{)}
\end{Verbatim}
\end{tcolorbox}

            \begin{tcolorbox}[breakable, size=fbox, boxrule=.5pt, pad at break*=1mm, opacityfill=0]
\prompt{Out}{outcolor}{38}{\boxspacing}
\begin{Verbatim}[commandchars=\\\{\}]
  param\_learning\_rate param\_max\_depth  mean\_test\_score  std\_test\_score
0                   0               1         0.581699        0.013072
1                   0               4         0.581699        0.013072
2                   0               8         0.581699        0.013072
3                   0              16         0.581699        0.013072
\end{Verbatim}
\end{tcolorbox}
        
    \begin{tcolorbox}[breakable, size=fbox, boxrule=1pt, pad at break*=1mm,colback=cellbackground, colframe=cellborder]
\prompt{In}{incolor}{39}{\boxspacing}
\begin{Verbatim}[commandchars=\\\{\}]
\PY{n}{best\PYZus{}params} \PY{o}{=} \PY{n}{grid\PYZus{}xgb}\PY{o}{.}\PY{n}{best\PYZus{}params\PYZus{}}
\PY{n+nb}{print}\PY{p}{(}\PY{l+s+s2}{\PYZdq{}}\PY{l+s+s2}{Best Parameters:}\PY{l+s+s2}{\PYZdq{}}\PY{p}{,} \PY{n}{best\PYZus{}params}\PY{p}{)}

\PY{n}{best\PYZus{}xgb} \PY{o}{=} \PY{n}{xgboost}\PY{o}{.}\PY{n}{XGBClassifier}\PY{p}{(}\PY{o}{*}\PY{o}{*}\PY{n}{best\PYZus{}params}\PY{p}{,} \PY{n}{random\PYZus{}state}\PY{o}{=}\PY{l+m+mi}{1234}\PY{p}{)}
\PY{n}{best\PYZus{}xgb}\PY{o}{.}\PY{n}{fit}\PY{p}{(}\PY{n}{X\PYZus{}train}\PY{p}{,} \PY{n}{y\PYZus{}train}\PY{p}{)}
\end{Verbatim}
\end{tcolorbox}

    \begin{Verbatim}[commandchars=\\\{\}]
Best Parameters: \{'learning\_rate': 0, 'max\_depth': 1\}
    \end{Verbatim}

            \begin{tcolorbox}[breakable, size=fbox, boxrule=.5pt, pad at break*=1mm, opacityfill=0]
\prompt{Out}{outcolor}{39}{\boxspacing}
\begin{Verbatim}[commandchars=\\\{\}]
XGBClassifier(base\_score=None, booster=None, callbacks=None,
              colsample\_bylevel=None, colsample\_bynode=None,
              colsample\_bytree=None, device=None, early\_stopping\_rounds=None,
              enable\_categorical=False, eval\_metric=None, feature\_types=None,
              gamma=None, grow\_policy=None, importance\_type=None,
              interaction\_constraints=None, learning\_rate=0, max\_bin=None,
              max\_cat\_threshold=None, max\_cat\_to\_onehot=None,
              max\_delta\_step=None, max\_depth=1, max\_leaves=None,
              min\_child\_weight=None, missing=nan, monotone\_constraints=None,
              multi\_strategy=None, n\_estimators=None, n\_jobs=None,
              num\_parallel\_tree=None, random\_state=1234, {\ldots})
\end{Verbatim}
\end{tcolorbox}
        
    The accuracy and performance of XGBoost is as follows.

    \begin{tcolorbox}[breakable, size=fbox, boxrule=1pt, pad at break*=1mm,colback=cellbackground, colframe=cellborder]
\prompt{In}{incolor}{40}{\boxspacing}
\begin{Verbatim}[commandchars=\\\{\}]
\PY{n}{y\PYZus{}pred} \PY{o}{=} \PY{n}{best\PYZus{}xgb}\PY{o}{.}\PY{n}{predict}\PY{p}{(}\PY{n}{X\PYZus{}test}\PY{p}{)}
\PY{n}{accuracy\PYZus{}score}\PY{p}{(}\PY{n}{y\PYZus{}pred}\PY{p}{,} \PY{n}{y\PYZus{}test}\PY{p}{)}
\end{Verbatim}
\end{tcolorbox}

            \begin{tcolorbox}[breakable, size=fbox, boxrule=.5pt, pad at break*=1mm, opacityfill=0]
\prompt{Out}{outcolor}{40}{\boxspacing}
\begin{Verbatim}[commandchars=\\\{\}]
0.5581395348837209
\end{Verbatim}
\end{tcolorbox}
        
    \begin{tcolorbox}[breakable, size=fbox, boxrule=1pt, pad at break*=1mm,colback=cellbackground, colframe=cellborder]
\prompt{In}{incolor}{41}{\boxspacing}
\begin{Verbatim}[commandchars=\\\{\}]
\PY{n}{confusion\PYZus{}mtx} \PY{o}{=} \PY{n}{confusion\PYZus{}matrix}\PY{p}{(}\PY{n}{y\PYZus{}test}\PY{p}{,} \PY{n}{y\PYZus{}pred}\PY{p}{)} 
\PY{n}{disp} \PY{o}{=} \PY{n}{ConfusionMatrixDisplay}\PY{p}{(}\PY{n}{confusion\PYZus{}matrix}\PY{o}{=} \PY{n}{confusion\PYZus{}mtx}\PY{p}{,} \PY{n}{display\PYZus{}labels}\PY{o}{=} \PY{k+kc}{None}\PY{p}{)}
\PY{n}{disp}\PY{o}{.}\PY{n}{plot}\PY{p}{(}\PY{n}{cmap}\PY{o}{=} \PY{n}{plt}\PY{o}{.}\PY{n}{cm}\PY{o}{.}\PY{n}{Blues}\PY{p}{)}
\PY{n}{plt}\PY{o}{.}\PY{n}{title}\PY{p}{(}\PY{l+s+s2}{\PYZdq{}}\PY{l+s+s2}{Confusion matrix XGBoost}\PY{l+s+s2}{\PYZdq{}}\PY{p}{)}
\PY{n}{plt}\PY{o}{.}\PY{n}{show}\PY{p}{(}\PY{p}{)}
\end{Verbatim}
\end{tcolorbox}

    \begin{center}
    \adjustimage{max size={0.9\linewidth}{0.9\paperheight}}{output_55_0.png}
    \end{center}
    { \hspace*{\fill} \\}
    
    We can see how the gradient boosting always predicts the most common
class, and so has a large amount of Type 1 errors.

    \hypertarget{predictor-comparison}{%
\section{Predictor comparison}\label{predictor-comparison}}

    The accuracies of the three proposed models are the following:

    \begin{tcolorbox}[breakable, size=fbox, boxrule=1pt, pad at break*=1mm,colback=cellbackground, colframe=cellborder]
\prompt{In}{incolor}{45}{\boxspacing}
\begin{Verbatim}[commandchars=\\\{\}]
\PY{n+nb}{print}\PY{p}{(}\PY{l+s+s2}{\PYZdq{}}\PY{l+s+s2}{Accuracy of RF:}\PY{l+s+s2}{\PYZdq{}}\PY{p}{,} \PY{n}{accuracy\PYZus{}score}\PY{p}{(}\PY{n}{best\PYZus{}rf}\PY{o}{.}\PY{n}{predict}\PY{p}{(}\PY{n}{X\PYZus{}test}\PY{p}{)}\PY{p}{,} \PY{n}{y\PYZus{}test}\PY{p}{)}\PY{p}{)}
\PY{n+nb}{print}\PY{p}{(}\PY{l+s+s2}{\PYZdq{}}\PY{l+s+s2}{Accuracy of GB:}\PY{l+s+s2}{\PYZdq{}}\PY{p}{,} \PY{n}{accuracy\PYZus{}score}\PY{p}{(}\PY{n}{best\PYZus{}gb}\PY{o}{.}\PY{n}{predict}\PY{p}{(}\PY{n}{X\PYZus{}test}\PY{p}{)}\PY{p}{,} \PY{n}{y\PYZus{}test}\PY{p}{)}\PY{p}{)}
\PY{n+nb}{print}\PY{p}{(}\PY{l+s+s2}{\PYZdq{}}\PY{l+s+s2}{Accuracy of XGB:}\PY{l+s+s2}{\PYZdq{}}\PY{p}{,} \PY{n}{accuracy\PYZus{}score}\PY{p}{(}\PY{n}{best\PYZus{}xgb}\PY{o}{.}\PY{n}{predict}\PY{p}{(}\PY{n}{X\PYZus{}test}\PY{p}{)}\PY{p}{,} \PY{n}{y\PYZus{}test}\PY{p}{)}\PY{p}{)}
\end{Verbatim}
\end{tcolorbox}

    \begin{Verbatim}[commandchars=\\\{\}]
Accuracy of RF: 0.5348837209302325
Accuracy of GB: 0.627906976744186
Accuracy of XGB: 0.5581395348837209
    \end{Verbatim}

    We can see that, in general, they're not very high. This can be due to
the number of NaNs in the data used. However, the Gradient Boosting
method obtains a result considerably better than the other two methods,
followed by the XGB and then the RF, which obtain similar results. For
this reason, the classifier we would choose would be Gradient Boosting.

    Using Gradient Boosting, the ten most relevant recurrence biomarkers are
the following peptides:

    \begin{tcolorbox}[breakable, size=fbox, boxrule=1pt, pad at break*=1mm,colback=cellbackground, colframe=cellborder]
\prompt{In}{incolor}{46}{\boxspacing}
\begin{Verbatim}[commandchars=\\\{\}]
\PY{n}{feature\PYZus{}importances} \PY{o}{=} \PY{n}{best\PYZus{}gb}\PY{o}{.}\PY{n}{feature\PYZus{}importances\PYZus{}}
\PY{n}{importance\PYZus{}gb} \PY{o}{=} \PY{n}{pd}\PY{o}{.}\PY{n}{DataFrame}\PY{p}{(}\PY{p}{\PYZob{}}\PY{l+s+s1}{\PYZsq{}}\PY{l+s+s1}{Feature}\PY{l+s+s1}{\PYZsq{}} \PY{p}{:} \PY{n}{X\PYZus{}train}\PY{o}{.}\PY{n}{columns}\PY{p}{,} \PY{l+s+s1}{\PYZsq{}}\PY{l+s+s1}{Importance}\PY{l+s+s1}{\PYZsq{}} \PY{p}{:} \PY{n}{feature\PYZus{}importances}\PY{p}{\PYZcb{}}\PY{p}{)}
\PY{n+nb}{print}\PY{p}{(}\PY{n}{importance\PYZus{}gb}\PY{o}{.}\PY{n}{sort\PYZus{}values}\PY{p}{(}\PY{n}{by}\PY{o}{=}\PY{p}{[}\PY{l+s+s1}{\PYZsq{}}\PY{l+s+s1}{Importance}\PY{l+s+s1}{\PYZsq{}}\PY{p}{]}\PY{p}{,} \PY{n}{ascending}\PY{o}{=}\PY{k+kc}{False}\PY{p}{)}\PY{o}{.}\PY{n}{head}\PY{p}{(}\PY{l+m+mi}{10}\PY{p}{)}\PY{p}{)}
\end{Verbatim}
\end{tcolorbox}

    \begin{Verbatim}[commandchars=\\\{\}]
    Feature  Importance
51       51    0.123774
11       11    0.112551
65       65    0.089577
60       60    0.067424
38       38    0.067323
70       70    0.067106
76       76    0.058383
19       19    0.048530
75       75    0.036543
67       67    0.032089
    \end{Verbatim}


    % Add a bibliography block to the postdoc
    
    
    
\end{document}
